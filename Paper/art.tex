\documentclass[acmtog, nonacm]{acmart}

\begin{document}

\title{State of the ART: An Index Structure for Main Memory Databases}

\author{Jonas Fritsch}
\affiliation{
    \institution{Technische Universtität München}
    \department{Fakultät für Informatik}}
\email{fritsch@in.tum.de}

\begin{abstract}
    With recent trends in hardware, main memory capacities have grown to an extent where most traditional DBMS 
    can fit entirely into main memory. These changes introduced a new shift of the performance bottleneck 
    from disk-based I/O to Main-Memory access.
    
    While previous Index-Structure like the B-Tree were optimized for minimizing disk access, the 
    adaptive radix tree (ART) is a trie-based index structure designed explicitly for in-memory usage. 
    It utilizes newer architecture features like SIMD and caching effectively and compresses its structure 
    dynamically, both horizontal and vertical. With these measures, ART achieves a performance that beats 
    other state-of-the-art order-preserving index structures in both insertion and single-lookup time.
\end{abstract}

\maketitle

\section{Introduction}
The architecture of DBMS has constantly been evolving due to advances in hardware. 
Over the last few decades, main memory capacities increased from several megabytes up to thousands 
of gigabytes, such that nowadays, databases can fit entirely into main memory. This change significantly 
impacted the general architecture of DBMS, which resulted in performance improvements 
by several factors \cite{harizopoulos2018oltp}, \cite{zhang2015memory}.

The design of index structures used to query a set of data more efficiently was heavily influenced by 
the main performance bottleneck of disk I/O in traditional disk-based DBMS.

\section{Adaptive Radix Tree}
Lorem ipsum.

\section{Constructing Binary-Comparable Keys}
Lorem ipsum.

\section{Evaluation}
Lorem ipsum.

\section{Related Work}
Lorem ipsum.

\section{Conclusion and Future Work}
Lorem ipsum.

\bibliographystyle{ACM-Reference-Format.bst}
\bibliography{refs}

\end{document}
\endinput