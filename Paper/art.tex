\documentclass[acmtog, nonacm]{acmart}

\begin{document}

\title{The State of the ART: An Index Structure for Main Memory Databases}

\author{Jonas Fritsch}
\affiliation{
    \institution{Technische Universtität München}
    \department{Fakultät für Informatik}}
\email{fritsch@in.tum.de}

\begin{abstract}
    With recent trends in hardware, main memory capacities have grown to an extent where most traditional DBMS 
    can fit entirely into main memory. These changes introduced a new shift of the performance bottleneck 
    from disk-based I/O to Main-Memory access.
    
    While previous Index-Structure like the B-Tree were optimized for minimizing disk access, the 
    adaptive radix tree (ART) is a trie-based index structure designed explicitly for in-memory usage. 
    It utilizes newer architecture features like SIMD and caching effectively and compresses its structure 
    dynamically, both horizontal and vertical. With these measures, ART achieves a performance that beats 
    other state-of-the-art order-preserving index structures in both insertion and single-lookup time.
\end{abstract}

\maketitle

\section{Introduction}
The architecture of DBMS has constantly been evolving due to advances in hardware. 
Over the last few decades, main memory capacities increased from several megabytes up to thousands 
of gigabytes, such that nowadays, databases can fit entirely into main memory. This change significantly 
impacted the general architecture of DBMS, which resulted in performance improvements 
by several factors \cite{10.1145/1376616.1376713}, \cite{7097722}.

The design of index structures used to query a set of data more efficiently was heavily influenced 
by the main performance bottleneck of disk I/O in traditional disk-based DBMS. 
Original index structures like the B-Tree designed to minimize disk accesses perform poorly 
in an in-memory environment. 

The T-Tree \cite{lehman1985study} was one of the first index structures proposed for main memory DBMS. 
However, over the last 35 years, the hardware landscape changed dramatically, causing T-Trees 
and all other index structures not explicitly designed with caching effects in mind to be rendered 
inefficient. Further focus on developing cache-sensitive index structures resulted in many 
different search tree variants.

Cache-sensitive search trees (CSS-Trees) \cite{rao1998cache}, while utilizing cache lines efficiently, 
introduce a significant overhead for updates as the tree is compactly stored in an array. 
Similarly, the more recent k-ary search tree \cite{10.1145/1565694.1565705} and the Fast Architecture 
Sensitive Tree (FAST) \cite{10.1145/1807167.1807206} both utilize Single Instruction Multiple Data (SIMD) 
instructions for data-level parallelism (DLP) to increase performance. However, as static data 
structures, they do not support incremental updates. A way to circumvent this limitation is to 
use a delta mechanism where another data structure stores differences and is periodically merged 
into the static structure. This comes at an additional performance cost. The cache-conscious 
B\textsuperscript{+}-Tree (CSB\textsuperscript{+}) \cite{10.1145/342009.335449} introduced as a variant of 
B\textsuperscript{+}-Trees improves cache utilization by reducing the need to store all different 
child pointers in each node.

Hash-Tables have been a popular indexing choice for main memory databases as they provide 
optimal $O(1)$ - as opposed to $O(log n)$ for search trees - single-lookup and update 
time on average. Many different hashing schemes and hash functions have been developed 
over time, but Hash-Tables generally do not support any range-based queries due to the nature 
of hash functions. Additionally, Hash-Tables can require complete re-hashing with $O(n)$ complexity 
upon reaching its load balance.

\section{Adaptive Radix Tree}
Lorem ipsum.

\section{Constructing Binary-Comparable Keys}
Lorem ipsum.

\section{Evaluation}
Lorem ipsum.

\section{Related Work}
Lorem ipsum.

\section{Conclusion and Future Work}
Lorem ipsum.

\bibliographystyle{ACM-Reference-Format.bst}
\bibliography{refs}

\end{document}
\endinput